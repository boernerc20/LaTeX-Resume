% AUTHOR: Christopher Boerner
% CREATE DATE: 1/8/25
% PURPOSE: LaTex version of resume.
% SPECIAL NOTES: Adapted from Hunter Ellis' resume (https://github.com/hunterwellis/LaTeX-Resume)

\documentclass[9pt, letterpaper]{extarticle}

% Packages:
\usepackage[
    ignoreheadfoot, % set margins without considering header and footer
    top=1.0 cm, % seperation between body and page edge from the top
    bottom=1.0 cm, % seperation between body and page edge from the bottom
    left=1.0 cm, % seperation between body and page edge from the left
    right=1.0 cm, % seperation between body and page edge from the right
    footskip=1.0 cm, % seperation between body and footer
    % showframe % for debugging 
]{geometry} % for adjusting page geometry
\usepackage{titlesec} % for customizing section titles
\usepackage{tabularx} % for making tables with fixed width columns
\usepackage{array} % tabularx requires this
\usepackage[dvipsnames]{xcolor} % for coloring text
\definecolor{primaryColor}{RGB}{0, 0, 0} % define primary color
\usepackage{enumitem} % for customizing lists
\usepackage{amsmath} % for math
\usepackage[
    pdftitle={Christopher Boerner's Resume},
    pdfauthor={Christopher Boerner},
    pdfcreator={LaTeX with VSCode},
    colorlinks=true,
    urlcolor=primaryColor
]{hyperref} % for links, metadata and bookmarks
\usepackage[pscoord]{eso-pic} % for floating text on the page
\usepackage{calc} % for calculating lengths
\usepackage{bookmark} % for bookmarks
\usepackage{lastpage} % for getting the total number of pages
\usepackage{changepage} % for one column entries (adjustwidth environment)
\usepackage{paracol} % for two and three column entries
\usepackage{ifthen} % for conditional statements
\usepackage{needspace} % for avoiding page brake right after the section title
\usepackage{iftex} % check if engine is pdflatex, xetex or luatex
\usepackage{datetime}

% Ensure that generate pdf is machine readable/ATS parsable:
\ifPDFTeX
    \input{glyphtounicode}
    \pdfgentounicode=1
    \usepackage[T1]{fontenc}
    \usepackage[utf8]{inputenc}
    \usepackage{lmodern}
\fi

\usepackage{charter}

% Some settings:
\raggedright
\AtBeginEnvironment{adjustwidth}{\partopsep0pt} % remove space before adjustwidth environment
\pagestyle{empty} % no header or footer
\setcounter{secnumdepth}{0} % no section numbering
\setlength{\parindent}{0pt} % no indentation
\setlength{\topskip}{0pt} % no top skip
\setlength{\columnsep}{0.15cm} % set column seperation
\pagenumbering{gobble} % no page numbering

\titleformat{\section}{\needspace{4\baselineskip}\bfseries\Large}{}{0pt}{}[\vspace{1pt}\titlerule]

\titlespacing{\section}{
    % left space:
    -1pt
}{
    % top space:
    0.15 cm
}{
    % bottom space:
    0.15 cm
} % section title spacing

\renewcommand\labelitemi{$\vcenter{\hbox{\small$\bullet$}}$} % custom bullet points
\newenvironment{highlights}{
    \begin{itemize}[
        topsep=0 pt,
        parsep=0 pt, 
        partopsep=0pt,
        itemsep=0pt,
        leftmargin=0.25 cm + 10pt
    ]
}{
    \end{itemize}
} % new environment for highlights


\newenvironment{highlightsforbulletentries}{
    \begin{itemize}[
        topsep=0.10 cm,
        parsep=0.10 cm,
        partopsep=0pt,
        itemsep=0pt,
        leftmargin=10pt
    ]
}{
    \end{itemize}
} % new environment for highlights for bullet entries

\newenvironment{onecolentry}{
    \begin{adjustwidth}{
        0 cm + 0.00001 cm
    }{
        0 cm + 0.00001 cm
    }
}{
    \end{adjustwidth}
} % new environment for one column entries

\newenvironment{twocolentry}[2][]{
    \onecolentry
    \def\secondColumn{#2}
    \setcolumnwidth{\fill, 4.5 cm}
    \begin{paracol}{2}
}{
    \switchcolumn \raggedleft \secondColumn
    \end{paracol}
    \endonecolentry
} % new environment for two column entries

\newenvironment{threecolentry}[3][]{
    \onecolentry
    \def\thirdColumn{#3}
    \setcolumnwidth{, \fill, \fill, \fill}
    \begin{paracol}{3}
    {\raggedright #2} \switchcolumn
    \begin{center}  % center the middle column content
}{
    \end{center}
    \switchcolumn \raggedleft \thirdColumn
    \end{paracol}
    \endonecolentry
} % new environment for three column entries

\newenvironment{header}{
    \setlength{\topsep}{0pt}\par\kern\topsep\linespread{1.5}
}{
    \par\kern\topsep
} % new environment for the header

% save the original href command in a new command:
\let\hrefWithoutArrow\href

% TITLE SECTION:
\begin{document}
    \newcommand{\AND}{\unskip
        \cleaders\copy\ANDbox\hskip\wd\ANDbox
        \ignorespaces
    }
    \newsavebox\ANDbox
    \sbox\ANDbox{$|$}
    \begin{threecolentry}
        {
            Falls Church, Virginia\\
            \hrefWithoutArrow{tel:+1-571-395-5448}{(571) 395-5448}\\
            \hrefWithoutArrow{mailto:boernerc20@gmail.com}{boernerc20@gmail.com}
        }
        {
            Active Secret Clearance \\
            Eagle Scout \\
            \hrefWithoutArrow{https://www.boernerc20.me}{boernerc20.me} \\
        }
        {
            \Huge\textbf{{Christopher Boerner}}\\
            \kern 3.0 pt%
            \Large{\textbf{Computer Engineer}}
        }
    \end{threecolentry}
    \hrule
    \kern 5.0 pt%
    Computer Engineering Master's graduate with experience in embedded systems, digital design, and computer-based control architectures. Skilled in integrating FPGA and microcontroller platforms with Ethernet and serial networks for real-time data acquisition. Strong background in systems analysis, hardware/software co-design, and validation of networked electronics for aerospace and power systems applications.

    % SKILLS SECTION:
    \section{Skills}
    \textbf{Languages: }C/C++, Python, Bash, Verilog, MATLAB\\
    \kern 5.0 pt%
    \textbf{Systems \& Networking: }Ethernet/TCP-IP, MQTT, Protocol stack, Real-time data exchange\\
    \kern 5.0 pt%
    \textbf{Software \& Tools: }Xilinx Vivado/Vitis, Alitum, STM32CubeIDE, KiCad, LTSpice, Cadence Virtuoso, Git, Linux, MATLAB, SolidWorks\\
    \kern 5.0 pt%
    \textbf{Hardware: }FPGA, Power System Integration, PCB layout, Circuit Analysis, EMI testing, UART/SPI/I2C interfacing\\
    
    % EDUCATION SECTION:
    \section{Education}
    \begin{twocolentry}{{May 2025}\\\textit{Alexandria, Virginia}}
        \textbf{Master of Engineering in Computer Engineering}\\
        Virginia Tech -- Focused on Computer Systems -- GPA: 3.8\\ 
        \quad\quad\textit{Advisor: \hrefWithoutArrow{https://www.yangyi.ece.vt.edu/}{{Dr. Cindy Yi (Virginia Tech)}}}\\
    \end{twocolentry}
    \kern 5.0 pt%
    \begin{twocolentry}{{May 2024}\\\textit{Blacksburg, Virginia}}
        \textbf{Bachelor of Science in Computer Engineering}\\
        Virginia Tech -- General Computer Engineering -- GPA: 3.6
    \end{twocolentry}
    
    % TECHNICAL EXPERIENCE SECTION:
    \section{Technical Experience}
    
    \begin{twocolentry}{{Nov 2024 -- May 2025} \\\textit{Alexandria, Virginia}}
        \textbf{Virginia Tech | FPGA-Accelerated Echo-State Network (Master’s Project)}\\
        \textbf{\textit{Graduate Researcher}}\\
        \begin{onecolentry}
            \begin{highlights}
                \item Designed an FPGA SoC implementing an Echo-State Network for real-time wireless radio channel prediction.
                \item Developed a TCP server for Ethernet-based communication between the Linux client and Zynq SoC.
                \item Implemented firmware modules for data buffering, fault handling, and timing verification across networked interfaces.
            \end{highlights}
        \end{onecolentry}
    \end{twocolentry}
    
    \kern 5.0 pt%
    \begin{twocolentry}{{Jun 2024 -- Aug 2024} \\\textit{Grenoble, France}}
        \textbf{Grenoble Electrical Engineering Laboratory | \hrefWithoutArrow{https://expe-smarthouse.org/index.php/en/project/}{Expe-SmartHouse Project}}\\
        \textbf{\textit{Research Intern}}\\
        \begin{onecolentry}
            \begin{highlights}
                \item Designed a distributed control system integrating photovoltaic energy sources and Wi-Fi connected microcontrollers.
                \item Programmed MQTT data exchange for power system coordination between houses and a centralized Raspberry Pi energy manager.
                \item Analyzed real-time energy data and network communication reliability.
            \end{highlights}
        \end{onecolentry}
    \end{twocolentry}
    
    \kern 5.0 pt%
    \begin{twocolentry}{{Aug 2023 -- May 2024} \\\textit{Blacksburg, Virginia}}
        \textbf{NAVAIR – Aircraft Data Acquisition System | Senior Design Project}\\
        \textbf{\textit{Project Member}}\\
        \begin{onecolentry}
            \begin{highlights}
                \item Developed a DAQ for aircraft diagnostics using STM32 and Artemis MCUs with UART and RF
                \item Designed PCB interfaces with power regulation, analog signal conditioning, and sensor isolation for vibration, temperature, sound, and humidity monitoring.
                \item Collaborated with my team on system integration, power distribution, and hardware verification.
            \end{highlights}
        \end{onecolentry}
    \end{twocolentry}
        
    \kern 5.0 pt%
    \begin{twocolentry}{{Sep 2022 -- May 2023} \\\textit{Blacksburg, Virginia}}
        \textbf{Systems Software Research Group | Computer Architecture Research}\\
        \textbf{\textit{Student Researcher}}\\
        \begin{onecolentry}
            \begin{highlights}
                \item Automated benchmark collection on FPGA-based RISC-V SoC using custom scripts.
                \item Modified instruction pipelines to study transient execution vulnerabilities.
            \end{highlights}
        \end{onecolentry}
    \end{twocolentry}
    \kern 5.0 pt%

\section{Projects}

\begin{minipage}[t]{0.48\textwidth}
\textbf{\large Academic \& Course Projects}

\vspace{4pt}

\textbf{VLSI Design Project | 12-bit Multiplier in Cadence Virtuoso} \\
\textit{Nov 2023 -- Dec 2023}
\begin{highlights}
    \item Designed a 12-bit Braun multiplier (schematic/layout) with carry-select adders in Cadence Virtuoso
    \item Verified functionality through DRC/LVS/PEX checks and measured propagation delay, power, and area for ADP optimization.  
\end{highlights}

\vspace{6pt}

\textbf{Integrated Design Project | Blood Oxygen Sensor} \\
\textit{Jan 2022 -- May 2022}
\begin{highlights}
    \item Created a multi-stage amplification and filtration circuit for photodiode sensor signals.
    \item Multiplexed between conditioned red and infrared channels to calculate blood oxygen saturation.
\end{highlights}
\end{minipage}
\hfill
\begin{minipage}[t]{0.48\textwidth}
\textbf{\large Personal Projects}

\vspace{4pt}

\textbf{Minecraft Jukebox Replica} \\
\textit{Jun 2025 -- Sep 2025}
\begin{highlights}
    \item Designed a proto-board for RFID sensing, power delivery, and analog audio output.
    \item 3D-printed enclosure and RFID-tagged discs trigger sound playback via ESP32 firmware.
\end{highlights}

\vspace{6pt}

\textbf{FPV Drone Design and Build} \\
\textit{May 2023 -- Present}
\begin{highlights}
    \item Built and tuned a 5" FPV drone with GPS telemetry, flight controller, and fail-safe power system.
    \item Configured radio link (ELRS) and PID loops for stable, high-performance flight.
\end{highlights}
\end{minipage}


\end{document}
